\glosario{

CMS: Un Sistema de gestión de contenidos (Content Management System en inglés, abreviado CMS) es un programa que permite crear una estructura de soporte (framework) para la creación y administración de contenidos, principalmente en páginas web, por parte de los participantes. Consiste en una interfaz que controla una o varias bases de datos donde se aloja el contenido del sitio. El sistema permite manejar de manera independiente el contenido y el diseño. Así, es posible manejar el contenido y darle en cualquier momento un diseño distinto al sitio sin tener que darle formato al contenido de nuevo, además de permitir la fácil y controlada publicación en el sitio a varios editores.

Copyleft: Es una forma de licencia y puede ser usado para modificar el derecho de autor de obras o trabajos, tales como software de computadoras, documentos, música, y obras de arte. Bajo tales licencias pueden protegerse una gran diversidad de obras, tales como programas informáticos, arte, cultura y ciencia, es decir prácticamente casi cualquier tipo de producción creativa.

Creative Commons: Es una organización no gubernamental sin ánimo de lucro que desarrolla planes para ayudar a reducir las barreras legales de la creatividad, por medio de nueva legislación y nuevas tecnologías. Fue fundada por Lawrence Lessig, profesor de derecho en la Universidad de Stanford y especialista en ciberderecho, que la presidió hasta marzo de 2008.

Distribución GNU/Linux: Versión de GNU/Linux que agrupa una serie de programas, que es organizada y mantenida adelante por algún grupo, empresa o institución.

Drupal: Es un programa de código abierto, con licencia GNU/GPL, escrito en PHP, desarrollado y mantenido por una activa comunidad de usuarios. Destaca por la cálidad de su código y de las páginas generadas, el respeto de los estándares de la web, y un énfasis especial en la usabilidad y consistencia de todo el sistema.

RSS: Es el acrónimo de Really Simple Syndication, una familia de formatos utilizada para publicar frecuentemente contenidos actualizados, como entradas de blogs o titulares de noticias. Un documento RSS puede contener un resumen de su contenido de un sitio web asociado o su texto completo.

}